\logosection{\faGraduationCap}{教育经历}

\datedline{\textbf{浙江大学}}{\dateRange{2022.7}{至今}}
\datedline{\tripleInfo{计算机科学与技术}{本科大三}{计算机科学与技术学院}}{浙江杭州}

\begin{itemize}
  \item GPA: 4.52/5.0
  \item 排名:37/169
  \item 重要课程成绩:操作系统 95 \ \ B/S体系软件设计 93  \ \ 计算机体系结构 92 \ \ 数据库系统 91 \ \ 数据结构 88
\end{itemize}

\logosection{\faHeart}{获奖情况}
\datedline{浙江大学二等奖学金}{2022-2023}
\datedline{浙江大学三等奖学金}{2023-2024}
\datedline{华为Hackathon软件难题挑战赛三等奖}{2024.10}
\datedline{华为软件精英挑战赛全国季军}{2025.04}


\logosection{\faCogs}{专业技能}
\begin{itemize}[parsep=0.5ex]
  \item 比较熟悉的编程语言: Python \ \  CPP 
  \item git,linux, anconda
  \item 对Pytorch有基本的了解,配过Pytorch环境,阅读过Pytorch实现的神经网络相关项目代码
  \item 学习过机器学习与深度学习相关的课程,对该领域基本概念有一定了解
\end{itemize}

\logosection{\faWrench}{项目经历}
\datedline{\textbf{Minisql-小型数据库}}{\dateRange{2024.03}{2024.05}}
\begin{itemize}
  \item 项目介绍:精简型单用户SQL引擎MiniSQL,允许用户通过字符界面输入SQL语句实现基本的增删改查操作
  \item 优化点1:增添索引功能,优化数据库增删改查性能
  \item 优化点2:添加新的缓冲区替换算法Clock Replacer,提高cache效率与能力
  \item 优化点3:设计堆表数据类型,优化记录在内存中的存储
\end{itemize}
\datedline{\textbf{RISC-V 操作系统内核实践}}{\dateRange{2024.09}{2024.11}}
\begin{itemize}
  \item 项目介绍:一个支持 RISC-V 架构的微型操作系统内核,专注于核心功能的实践,包括虚拟内存、进程管理和文件系统
  \item 使用 RISC-V 汇编完成底层引导与初始化,包括设置初始页表、中断向量表,并配置关键控制状态寄存器
  \item 独立开发了基础进程管理模块,包含进程上下文、虚拟内存空间及页表的复制
  \item 构建了完整的系统调用处理流程,提供了标准的文件和进程操作接口
\end{itemize}
\datedline{\textbf{商品比价系统}}{\dateRange{2024.11}{2024.11}}
\begin{itemize}
  \item 项目介绍:独立开发并部署了一个全栈Web应用,旨在解决用户跨电商平台(淘宝、京东)比价和追踪商品价格的需求。系统允许用户搜索商品、查看历史价格、订阅感兴趣的商品,并在商品降价时自动发送邮件通知,提升购物决策效率
  \item 技术栈:python flask, vue3, DrissionPage爬虫,Docker打包
\end{itemize}
\datedline{\textbf{华为软件精英挑战赛 - 面向对象存储读写算法设计与实现 (C++)}}{\dateRange{2025.03}{2025.04}}
\begin{itemize}
    \item 项目介绍:基于C++实现对象存储系统核心算法
    \item 1. 高效读写调度(结合DP成本计算与请求价值)
    \item 2. 智能三副本写入(平衡冲突与空间预留)
    \item 3. 基于多维评分(碎片化、价值、空间利用率)的垃圾回收机制
    \item 4. 标签预测提升决策精度
\end{itemize}

\logosection{\faWrench}{科研经历}
\datedline{\textbf{SRTP 面向药物相互作用预测任务的系统设计与实现(校级)}}{\dateRange{2024}{2025.05}}
\begin{itemize}
    \item 项目介绍:图神经网络架构,双视角视图,小分子药物与蛋白质药物,互作用关系预测,可视化平台构建
    \item github项目地址:https://github.com/ZhengliangDuanfang/ourCBTIP
\end{itemize}

\datedline{\textbf{脑机接口算法原理与实践}}{\dateRange{2024.7}{2024.7}}
在这门短学期课中,我们体验了脑机接口领域基本算法,尝试解决该领域相关问题:左右手想象识别,情绪识别等。自己使用 Pytorch 处理脑电波数据构建数据集,搭建神经网络,训练模型,得到预测结果。

\datedline{\textbf{基于GRPO的坐标序列预测}}{\dateRange{2025.05}{2025.06}}
\begin{itemize}
    \item 项目介绍:基于GRPO强化学习算法和Qwen2.5-1.5B-Instruct模型,实现一维匀速直线运动的坐标序列预测系统
    \item 核心工作:
        \begin{itemize}
            \item 设计并实现了准确性奖励和格式奖励两个奖励函数,采用5:1的权重比优化预测效果
            \item 通过Few-shot和Chain-of-thought提示工程,提升模型的推理能力和输出质量
            \item 构建了完整的评估体系,实现90\%预测序列误差控制在0.2以内
        \end{itemize}
    \item 技术要点:Python,PyTorch,HuggingFace,强化学习,大语言模型
\end{itemize}


